\documentclass[12pt]{article}
\usepackage{graphicx}

%Following 2 lines were added to remove the blank page at the beginning
\usepackage{atbegshi}% http://ctan.org/pkg/atbegshi
\AtBeginDocument{\AtBeginShipoutNext{\AtBeginShipoutDiscard}}
%

\begin{document}

\begin{center}
\title{\textbf{Area of a Traingle}}
\date{\vspace{-5ex}} %Not to print date automatically
\maketitle
\end{center}

\setcounter{page}{1}

\section{10$^{th}$ Maths - Chapter 7}

All problems are from Exercise 7.3

\begin{enumerate}
\item Find the area of the triangle whose vertices are :
\begin{enumerate}
\item ((2, 3), (–1, 0), (2, – 4)
\item (–5, –1), (3, –5), (5, 2) 
\end{enumerate}

\item In each of the following, find the value of '\textit{k}', for which the points are collinear.
\begin{enumerate}
\item (7, –2), (5, 1), (3, \textit{k})
\item (8, 1), (\textit{k}, – 4), (2, –5) 
\end{enumerate}

\item Find the area of the triangle formed by joining the mid-points of the sides of the triangle whose vertices are (0, –1), (2, 1) and (0, 3). Find the ratio of this area to the area of the given triangle.

\item Find the area of the quadrilateral whose vertices, taken in order, are (– 4, – 2), (– 3, – 5),(3, – 2) and (2, 3).

\item You have studied in Class IX, (Chapter 9, Example 3), that a median of a triangle divides it into two triangles of equal areas. Verify this result for $\triangle$ABC whose vertices are $\vec{A}$(4, – 6), $\vec{B}$(3, –2) and $\vec{C}$(5, 2). 

\end{enumerate}

\section{12$^{th}$ Maths - Chapter 8} % * indicates no numbering for section
\begin{enumerate}
\item Using integration find the area of region bounded by the triangle whose
vertices are (1, 0), (2, 2) and (3, 1) (Ref : Example 9)

\begin{figure}[!h]
	\includegraphics[width=10cm,height=7cm]{./fig1}
\label{fig:Fig1}
\end{figure}

\item Using integration find the area of region bounded by the triangle whose vertices
are (– 1, 0), (1, 3) and (3, 2). (Ref : Problem 4 in Ex 8.2)

\item Using the method of integration find the area of the $\triangle$ABC, coordinates of whose vertices are $\vec{A}$(2, 0), $\vec{B}$(4, 5) and $\vec{C}$(6, 3). (Ref: Problem 13 in Misc Exercise on Chapter 8)

\end{enumerate}
\end{document}

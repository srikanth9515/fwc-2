\documentclass[12pt]{article}
\usepackage{graphicx}

%Following 2 lines were added to remove the blank page at the beginning
\usepackage{atbegshi}% http://ctan.org/pkg/atbegshi
\AtBeginDocument{\AtBeginShipoutNext{\AtBeginShipoutDiscard}}
%

\begin{document}

\begin{center}
\title{\textbf{CHAPTER-7} \\Area of a Traingle}
\date{\vspace{-5ex}} %Not to print date automatically
\maketitle
\end{center}

\section*{Exercise 7.3} % * indicates no numbering for section
\begin{enumerate}

\item Find the area of the triangle whose vertices are : \\
(i) (2, 3), (–1, 0), (2, – 4) \hspace{8mm}      (ii) (–5, –1), (3, –5), (5, 2) 

\vspace{3mm}

\item In each of the following, find the value of '\textit{k}', for which the points are collinear. \\
(i) (7, –2), (5, 1), (3, \textit{k})   \hspace{10mm}       (ii) (8, 1), (\textit{k}, – 4), (2, –5) 

\vspace{3mm}

\item Find the area of the triangle formed by joining the mid-points of the sides of the triangle whose vertices are (0, –1), (2, 1) and (0, 3). Find the ratio of this area to the area of the given triangle.
\vspace{3mm}

\item Find the area of the quadrilateral whose vertices, taken in order, are (– 4, – 2), (– 3, – 5),(3, – 2) and (2, 3).

\vspace{3mm}

\item You have studied in Class IX, (Chapter 9, Example 3), that a median of a triangle divides it into two triangles of equal areas. Verify this result for $\triangle$\textbf{ABC} whose vertices are \textbf{A}(4, – 6), \textbf{B}(3, –2) and \textbf{C}(5, 2). 

\end{enumerate}

\end{document}
